\HeaderA{kin-class}{Class "kin" for kinetic model storage.}{kin.Rdash.class}
\aliasA{kin}{kin-class}{kin}
\keyword{classes}{kin-class}
\begin{Description}\relax
\code{kin} is the class for kinetic models; if \code{mod\_type = "kin"} is an 
argument of \code{initModel}.
All objects of class \code{kin} are sub-classes of 
class \code{dat}; see documentation for \code{dat} 
for a description of 
these slots.
\end{Description}
\begin{Details}\relax
See \code{\LinkA{dat-class}{dat.Rdash.class}} for an 
example of the initialization of a 
\code{kin} object via the \code{initModel} function.
\end{Details}
\begin{Section}{Objects from the Class}
Objects can be created by calls of the form \code{new("kin", ...)} or 
\code{kin(...)}.
\end{Section}
\begin{Section}{Slots}
\describe{
\item[kinpar] vector of rate constants to be used as starting 
values for the exponential decay of components; the length of this
vector determines the number of components of the kinetic model.                        
}
\item[\code{specpar}:] Object of class \code{"list"}  parameters for
spectral constraints
\item[\code{seqmod}:] Object of class \code{"logical"}  that is \code{TRUE} if a sequential model is to be applied
and \code{FALSE} otherwise
\item[\code{irf}:] Object of class \code{"logical"}   that is \code{TRUE} is an IRF is modeled and \code{FALSE} otherwise
\item[\code{mirf}:] Object of class \code{"logical"}  that is \code{TRUE} if a measured IRF is modeled and \code{FALSE}
otherwise
\item[\code{measured\_irf}:] Object of class \code{"vector"}  containing a measured IRF
\item[\code{convalg}:] Object of class \code{"numeric"}  1-4 determining the numerical convolution algorithm 
used in the case of modeling a measured IRF
\item[\code{irffun}:] Object of class \code{"character"}  describing the 
function to use to describe the IRF, by default "gaus"
\item[\code{irfpar}:] Object of class \code{"vector"}  of IRF parameters; for the common Gaussian IRF this 
vector is ordered \code{c(location, width)}
\item[\code{dispmu}:] Object of class \code{"logical"}   that is \code{TRUE} if dispersion of the parameter for IRF 
location is to be modeled and \code{FALSE} otherwise
\item[\code{dispmufun}:] Object of class \code{"character"}  describing the functional form of the 
dispersion of the IRF location parameter; if equal to "discrete" then the 
IRF location is shifted per element of \code{x2} and \code{parmu} should have the same 
length as \code{x2}.  defaults to a polynomial description
\item[\code{parmu}:] Object of class \code{"list"}   of starting values for the dispersion model for the 
IRF location
\item[\code{disptau}:] Object of class \code{"logical"}  that is \code{TRUE} if dispersion of the parameter for 
IRF width is to be modeled and \code{FALSE} otherwise
\item[\code{disptaufun}:] Object of class \code{"character"}  describing the functional form of the 
dispersion of the IRF width parameter; if equal to \code{"discrete"} then the 
IRF width is parameterized per element of \code{x2} and \code{partau} should have the same 
length as \code{x2}.  defaults to a polynomial description
\item[\code{partau}:] Object of class \code{"vector"}  of starting values for the dispersion model for the 
IRF FWHM 
\item[\code{fullk}:] Object of class \code{"logical"}  that is \code{TRUE} if the data are to be modeled using a 
compartmental model defined in a K matrix and \code{FALSE} otherwise
\item[\code{kmat}:] Object of class \code{"array"}  containing the K matrix descriptive of  a compartmental 
model
\item[\code{jvec}:] Object of class \code{"vector"}  containing the J vector descriptive of the inputs to a 
compartmental model
\item[\code{ncolc}:] Object of class \code{"vector"}  describing the number of columns of the C matrix for 
each clp in \code{x2}
\item[\code{kinscal}:] Object of class \code{"vector"}   of starting values for branching parameters in a 
compartmental model
\item[\code{kmatfit}:] Object of class \code{"array"}  of fitted values for a compartmental model
\item[\code{cohspec}:] Object of class \code{"list"}   describing the model for 
coherent artifact/scatter component(s) containing the element  \code{type}
and optionally the element \code{numdatasets}
if \code{type} is \code{"irf"}, the coherent artifact/scatter has the 
time profile of 
the IRF.  if \code{type} is \code{"freeirfdisp"} the  coherent 
artifact/scatter has  a
Gaussian time profile whose location and width are parameterized in the 
vector \code{coh}.  if \code{type} is \code{"irfmulti"} the time profile of 
the IRF is used for
the coherent artifact/scatter model, but the IRF parameters are taken per 
dataset (for the multidataset case), and the integer argument 
\code{numdatasets} must be equal to the 
number of datasets modeled.  if \code{type} is \code{"seq"} 
a sequential exponential decay 
model is applied, whose parameters are contained in \code{coh}.
if \code{type} is \code{"mix"} a sequential exponential decay 
model is applied along with a model that follows the time profile of the IRF;
the coherent artifact/scatter is then a linear superposition of these two 
models.  
\item[\code{coh}:] Object of class \code{"vector"}  of starting values for the parameterization of a  
coherent artifact
\item[\code{wavedep}:] Object of class \code{"logical"}  describing whether the kinetic model is dependent on
\code{x2} index (i.e., whether there is clp-dependence)
\item[\code{lambdac}:] Object of class \code{"numeric"}  for the center wavelength to be used in a polynomial 
description of \code{x2}-dependence 
\end{Section}
\begin{Section}{Extends}
Class \code{\LinkA{dat-class}{dat.Rdash.class}}, directly.
\end{Section}
\begin{Author}\relax
Katharine M. Mullen, Ivo H. M. van Stokkum
\end{Author}
\begin{SeeAlso}\relax
\code{\LinkA{dat-class}{dat.Rdash.class}}, \code{\LinkA{spec-class}{spec.Rdash.class}}
\end{SeeAlso}

