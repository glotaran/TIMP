\HeaderA{spec-class}{Class "spec" for the storage of spectral models.}{spec.Rdash.class}
\aliasA{spec}{spec-class}{spec}
\keyword{classes}{spec-class}
\begin{Description}\relax
\code{spec} is the class for spectral models; if \code{mod\_type = "spec"} 
is an 
input to \code{initModel}. All objects of class 
\code{spec} are
also of class \code{dat}; see documentation for \code{dat} 
for a description of 
these slots.  Note that here \code{x2} 
will refer to the independent variable in
which traces are resolved, e.g., wavelength or wavenumber.
\end{Description}
\begin{Section}{Objects from the Class}
Objects can be created by calls of the form \code{new("spec", ...)} or 
\code{spec(...)}.
\end{Section}
\begin{Section}{Slots}
\describe{
\item[\code{clpequ}:] Object of class \code{"vector"}  of starting values for linear relationships between 
clp
\item[\code{specpar}:] Object of class \code{"list"}  of vectors of 
starting values for spectral parameters; the number of vectors gives the 
number of components in the resulting spectral model;  
each vector contains the parameters 
associated with a component.  e.g., 
\code{specpar = list(c(20000, 3000, .3, 21000, 2000, .4), c(18000, 1000, .2))};
the parameters in each vector are grouped 
\code{c(location\_spectra, width\_spectra, skew\_spectra)}.  
the location and width parameters are given in wavenumbers.  

\item[\code{specfun}:] Object of class \code{"character"},  \code{"gaus"} for a spectral model of a
superposition of skewed Gaussians; \code{"bspline"} for a 
bspline-based model.  
\item[\code{specref}:] Object of class \code{"numeric"}  index defining the center value of the 
\code{x2} variable.
\item[\code{specCon}:] Object of class \code{"list"}  used internally to 
store constraints. 
\item[\code{specdisp}:] Object of class \code{"logical"}  \code{TRUE} if time-dependence of the spectral parameters
is to be taken into account and \code{FALSE} otherwise
\item[\code{specdisppar}:] Object of class \code{"list"}  
\item[\code{specdispindex}:] Object of class \code{"list"}  of vectors defining those indexes of specpar whose
time-dependence is to be modeled.  e.g., \code{specdispindex = list(c(1,1), 
c(1,2), c(1,3))} says that parameters 1-3 of spectra 1 are to be modeled as 
time-dependent.
\item[\code{nupow}:] Object of class \code{"numeric"}   describing the power to which wavenumbers are raised in 
the model equation; see Equation 30 of the paper in the references section 
for a complete description
\item[\code{timedep}:] Object of class \code{"logical"} describing whether the model for spectra E is dependent
on x-index (i.e., whether it is clp-dependent). 
\item[\code{parmufunc}:] Object of class \code{"character"}  describing the function form of the 
time-dependence of spectral parameters; options are \code{"exp"} 
for exponential time 
dependence, \code{"multiexp"} for multiexponential time dependence, and 
\code{"poly"} for 
polynomial time dependence. defaults to polynomial time dependence.  
\item[ ncole ] vector describing the number of columns of the E matrix for 
each value in the \code{x} vector
}
\end{Section}
\begin{Section}{Extends}
Class \code{\LinkA{dat-class}{dat.Rdash.class}}, directly.
\end{Section}
\begin{Author}\relax
Katharine M. Mullen, Ivo H. M. van Stokkum
\end{Author}
\begin{References}\relax
Ivo H. M. van Stokkum, 
"Global and target analysis of time-resolved spectra, Lecture notes
for the Troisieme Cycle de la Physique en Suisse Romande", 
Department of Physics and Astronomy, Faculty of Sciences, Vrije Universiteit,
Amsterdam, The Netherlands, 2005, 
\url{http://www.nat.vu.nl/~ivo/lecturenotes.pdf}
\end{References}
\begin{SeeAlso}\relax
\code{\LinkA{kin-class}{kin.Rdash.class}},
\code{\LinkA{dat-class}{dat.Rdash.class}}
\end{SeeAlso}

