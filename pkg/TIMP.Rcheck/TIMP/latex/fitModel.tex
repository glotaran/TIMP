\HeaderA{fitModel}{Performs optimization of (possibly multidataset) models.}{fitModel}
\keyword{file}{fitModel}
\begin{Description}\relax
Performs optimization of (possibly multidataset) models and 
outputs plots and files representing the fit of the model to the data.
\end{Description}
\begin{Usage}
\begin{verbatim}
fitModel(data, modspec=list(), datasetind = vector(), modeldiffs = list(), 
                opt = opt() )
\end{verbatim}
\end{Usage}
\begin{Arguments}
\begin{ldescription}
\item[\code{data}] list of data objects of class \code{dat}
\item[\code{modspec}] list whose elements are models of class \code{dat} 
describing the models as results from a
call to the function \code{initModel} 
\item[\code{datasetind}] vector that has the same length as \code{data};
for each dataset in \code{data} specify the model it should have as 
an index into \code{modspec}; default mapping is that all datasets 
use the first model given in \code{modspec}  
\item[\code{modeldiffs}] list whose elements specify any dataset-specific 
model differences. 
\Itemize{ 
\item[dscal] list of lists specifying linear scaling relations between 
datasets; each list has elements \code{to, from, value}.  The index of the 
dataset to be scaled is given in \code{to}; the index of the dataset on 
which the scaling is to be based is given in \code{from}.  The starting 
value parameterizing the relationship is given as \code{value}. For 
example, \code{dscal = list(list(to=2,from=1,value=.457))}.     
\item[thresh] numeric describing the tolerance with which clp from 
different datasets are to be considered as equal.  
For instance, for 
two datasets containing data at 750 and 751 nm, respectively, 
\code{thresh=1.5} will equate the clp at 750 and 751 between datasets.
Specify a negative
value of \code{thresh} to estimate clp per-dataset. 
See Section 2.2 of the paper in the references for the model equations.  
\item[free] list of lists specifying individual 
parameters to free for a given dataset. each sublist has 
named elements 
\Itemize{ 
\item[what] character string naming parameter type, e.g., "kinpar"
\item[ind] vector of indices into parameter vector or list, e.g.,
\code{c(2,3)} or \code{4}
\item[dataset] dataset index in which parameter is to be freed
\item[start] starting value for freed parameter
}  
For example, \code{free = list(
list(what = "irfpar", ind = 1, dataset = 2, start=-.1932),
list(what = "kinpar", ind = 5, dataset = 2, start=.0004),
list(what = "kinpar", ind = 4, dataset = 2, start= .0159))}. 

\item[remove] list of lists specifying individual 
parameters to remove from 
parameter groups for a given dataset. each sublist has 
named elements 
\Itemize{ 
\item[what] character string naming parameter type, e.g., "kinpar"
\item[dataset] dataset index in which parameter group is to be removed
\item[ind] vector of indices into parameter vector or list, e.g.,
\code{c(2,3)} or \code{4} where parameter should be removed 
}
  
\item[add] list of lists specifying individual 
parameters to add to parameter groups for a given dataset. each sublist has 
named elements 
\Itemize{ 
\item[what] character string naming parameter type, e.g., "kinpar"
\item[dataset] dataset index in which parameter group is to change
\item[start] starting value for added parameter
\item[ind] vector of indices into parameter vector or list, e.g.,
\code{c(2,3)} or \code{4} where parameter should be added. 
}
  
\item[change] list of lists specifying entire parameter groups to change 
for a given dataset.
each sublist has named elements 
\Itemize{ 
\item[what] character string naming parameter type, e.g., "kinpar"
\item[dataset] dataset index in which parameter group is to change
\item[spec] new specification that in initModel would follow "what", 
e.g., for \code{c(.1, .3)} if what="kinpar"
}
\item[rel] list of lists specifying parameters to relate between datasets
each sublist has named elements 
\Itemize{ 
\item[what1] character string naming parameter type to be determined in 
relation to some other parameter type , e.g., "kinpar"
\item[what2] character string naming parameter type on which another 
parameter type is to depend, e.g., "kinpar"
\item[ind1] vector of indices into parameter vector or 
list, e.g., \code{c(2,3)} or \code{4} of the dependent parameter. }
\item[ind2] vector or numeric of indices into parameter vector or 
list, e.g., \code{c(2,3)} or \code{4} of the parameter on which another 
parameter will depend 
\item[dataset1] dataset index of the dependent parameter
\item[dataset2] dataset index of the parameter on which another parameter
will depend
\item[rel] optional character string describing functional relationship 
between parameters; defaults to "lin" for linear relationship 
\item[start] starting value or vector of values parameterizing 
relationship between parameters  
 

}

\item[\code{opt}] Object of class \code{kinopt} or \code{specopt} specifying 
fitting and plotting options. 
\end{ldescription}
\end{Arguments}
\begin{Details}\relax
This function applies the \code{\LinkA{nls}{nls}} function internally to 
optimize nonlinear parameters and to solve for conditionally linear parameters
(clp) via the partitioned variable projection algorithm.
\end{Details}
\begin{Value}
list with element \code{toPlotter}.  
\begin{ldescription}
\item[\code{toPlotter}] is a list containing all arguments used by the plotting 
function; it is used to regenerate plots and other output by the 
\code{examineFit}  function
\end{ldescription}

normal-bracket116bracket-normal
\end{Value}
\begin{Author}\relax
Katharine M. Mullen, Ivo H. M. van Stokkum
\end{Author}
\begin{References}\relax
Mullen KM, van Stokkum IHM (2007). 
``TIMP: an R package for modeling
multi-way spectroscopic measurements.'' Journal of Statistical Software,
18(3). \url{http://www.jstatsoft.org/v18/i03/.}
\end{References}
\begin{SeeAlso}\relax
\code{\LinkA{readData}{readData}}, \code{\LinkA{initModel}{initModel}}, 
\code{\LinkA{examineFit}{examineFit}}
\end{SeeAlso}

