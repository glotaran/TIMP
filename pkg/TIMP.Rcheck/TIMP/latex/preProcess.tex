\HeaderA{preProcess}{Performs preprocessing on data stored as an objects of class dat.}{preProcess}
\keyword{file}{preProcess}
\begin{Description}\relax
Performs data sampling, selection, baseline correction,  
scaling, and data correction on an object of class \code{dat}.
\end{Description}
\begin{Usage}
\begin{verbatim}
preProcess(data, sample = 1, sample_time = 1, sample_lambda = 1, 
    sel_time = vector(), sel_lambda = vector(), baselinetime = vector(), 
    baselinelambda = vector(), scalx = NULL, scalx2 = NULL, 
    sel_lambda_ab = vector(), sel_time_ab = vector(), rm_x2=vector(), 
    rm_x = vector(), svdResid = list(), numV = 0)
\end{verbatim}
\end{Usage}
\begin{Arguments}
\begin{ldescription}
\item[\code{data}] Object of class \code{dat}
\item[\code{sample}] integer describing sampling interval to take in both time and 
\code{x2}; e.g., \code{sample=2} 
will sample every 2nd time and every 2nd point in 
\code{x2}.
\item[\code{sample\_time}] integer describing sampling interval in time; e.g.,
\code{sample\_time=2} will sample every 2nd element of the time vector. 
\item[\code{sample\_lambda}] integer describing sampling interval in \code{x2}; 
e.g., \code{sample\_lambda=2} will sample every 2nd element in the 
\code{x2} vector. 
\item[\code{sel\_time}] vector of length 2 describing the first and last time 
index of data to select; e.g., \code{sel\_time=c(5,120)} will select 
data at times indexed 5-120. 
\item[\code{sel\_lambda}] vector of length 2 describing the first and last \code{x2}  
index of data to select; e.g., \code{sel\_lambda=c(5,120)} 
will select data at \code{x2}
indexed 5-120. 
\item[\code{baselinetime}] a vector of form \code{c(timeIndexmin, timeIndexmax, 
  lambdaIndexmin, lambdaIndexmax)}.  The average of data between 
\code{x2} indexes  
\code{lambdaIndexmin} and \code{lambdaIndexmax} 
is subtracted from data with 
time index between \code{timeIndexmin} and \code{timeIndexmax}.  
\item[\code{baselinelambda}] a vector of form \code{c(timeIndexmin, timeIndexmax, 
  lambdaIndexmin, lambdaIndexmax)}.  The average of data between time indexes  
\code{timeIndexmin} and \code{timeIndexmax} 
is subtracted from data with \code{x2} index 
between \code{lambdaIndexmin} and \code{lambdaIndexmax}.
\item[\code{scalx}] numeric by which to linearly scale the \code{x} axis
(which often represents time), so that newx = oldx * scalx 
\item[\code{scalx2}] vector of length 2 by which to linearly scale the 
\code{x2} axis, so that newx2 = oldx2 * scalx2[1] + scalx2[2]
\item[\code{sel\_lambda\_ab}] vector of length 2 describing the absolute values
(e.g., wavelengths, wavenumbers, etc.) between which data should be
selected.   e.g., \code{sel\_lambda\_ab = c(400, 600)} will select data  
associated with \code{x2} values between 400 and 600. 
\item[\code{sel\_time\_ab}] vector of length 2 describing the absolute times 
between which data should be
selected.   e.g., \code{sel\_time\_ab = c(50, 5000)} will select data  
associated with time values between 50 and 5000 picoseconds.
\item[\code{rm\_x2}] vector of \code{x2} indices to remove from the data
\item[\code{rm\_x}] vector of \code{x} indices to remove from the data
\item[\code{svdResid}] list returned from the \code{getResid} function, containing
residuals to be used in data correction. 
\item[\code{numV}] numeric specifying how many singular vectors to use in data
correction.  Maximum is five. 
\end{ldescription}
\end{Arguments}
\begin{Value}
object of class \code{dat}.
\end{Value}
\begin{Author}\relax
Katharine M. Mullen, Ivo H. M. van Stokkum
\end{Author}
\begin{SeeAlso}\relax
\code{\LinkA{readData}{readData}}, \code{\LinkA{getResid}{getResid} }
\end{SeeAlso}

