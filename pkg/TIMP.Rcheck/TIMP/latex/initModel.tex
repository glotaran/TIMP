\HeaderA{initModel}{Defines the model to be used in analysis.}{initModel}
\keyword{file}{initModel}
\begin{Description}\relax
Allows definition of a model of class "dat" to be used 
in analysis.  The arguments specify the model.
\end{Description}
\begin{Usage}
\begin{verbatim}
initModel(...)
\end{verbatim}
\end{Usage}
\begin{Arguments}
\begin{ldescription}
\item[\code{...}] specify the model class via the character string
e.g., \code{\LinkA{kin-class}{kin.Rdash.class}} or \code{\LinkA{spec}{spec}} and 
any of the slots associated with that model type (which is 
a subclass of class \code{dat}, so that all slots in \code{dat} 
may also be specified),   
e.g., \code{mod\_type = "kin"} will initialize a model with 
class \code{kin}, for a kinetic model.  
\end{ldescription}
\end{Arguments}
\begin{Details}\relax
For examples, see the help files for \code{\LinkA{dat-class}{dat.Rdash.class}} and 
\code{\LinkA{fitModel}{fitModel}}
\end{Details}
\begin{Value}
an object of class \code{dat} with the sub-class given by the value of 
the \code{mod\_type} input.
\end{Value}
\begin{Author}\relax
Katharine M. Mullen, Ivo H. M. van Stokkum
\end{Author}
\begin{SeeAlso}\relax
\code{\LinkA{dat-class}{dat.Rdash.class}}, \code{\LinkA{kin-class}{kin.Rdash.class}}, 
\code{\LinkA{spec-class}{spec.Rdash.class}}, 
\code{\LinkA{fitModel}{fitModel}}
\end{SeeAlso}

