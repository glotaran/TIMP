\HeaderA{getClpindepX-methods}{Generic function getClpindepX in Package `TIMP'}{getClpindepX.Rdash.methods}
\aliasA{getClpindepX}{getClpindepX-methods}{getClpindepX}
\aliasA{getClpindepX,kin-method}{getClpindepX-methods}{getClpindepX,kin.Rdash.method}
\aliasA{getClpindepX,spec-method}{getClpindepX-methods}{getClpindepX,spec.Rdash.method}
\keyword{methods}{getClpindepX-methods}
\begin{Description}\relax
Gets the matrix associated with nonlinear parameter estimates for the 
case that this matrix is not re-calculated per conditionally linear
parameter.
\end{Description}
\begin{Usage}
\begin{verbatim}
getClpindepX(model, multimodel, theta, returnX, rawtheta, dind)
\end{verbatim}
\end{Usage}
\begin{Arguments}
\begin{ldescription}
\item[\code{model}] Object of class \code{dat}; function switches on this 
argument. 

\item[\code{multimodel}] Object of class \code{multimodel} used in standard error
determination
\item[\code{theta}] Vector of nonlinear parameter estimates.
\item[\code{returnX}] logical indicating whether to return a vectorized version of 
the \code{X} matrix
\item[\code{rawtheta}] vector of nonlinear parmeters; used in standard error
determination
\item[\code{dind}] numeric indicating the dataset index; used in standard error
determination
\end{ldescription}
\end{Arguments}
\begin{Author}\relax
Katharine M. Mullen, Ivo H. M. van Stokkum
\end{Author}
\begin{SeeAlso}\relax
\code{\LinkA{dat-class}{dat.Rdash.class}}
\end{SeeAlso}

