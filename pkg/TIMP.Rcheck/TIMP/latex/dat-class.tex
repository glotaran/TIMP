\HeaderA{dat-class}{Class "dat" for model and data storage}{dat.Rdash.class}
\aliasA{dat}{dat-class}{dat}
\keyword{classes}{dat-class}
\begin{Description}\relax
\code{dat} is the super-class of other classes representing models and data, so
that other model/data classes (e.g., \code{kin} and \code{spec} 
for kinetic and spectral
models respectively) also have the slots defined here.  These slots may
be specified in the \code{...} 
argument of the  \code{\LinkA{initModel}{initModel}} function.
\end{Description}
\begin{Section}{Objects from the Class}
Objects from the class 
can be created by calls of the form \code{new("dat", ...)} or 
\code{dat(...)}, but 
most are most often made by invoking another function such as 
\code{readData} or \code{initModel}.
\end{Section}
\begin{Section}{Slots}
\describe{
\item[\code{psi.df}:] Object of class \code{"matrix"}  dataset from 1 experiment
\item[\code{psi.weight}:] Object of class \code{"matrix"} weighted  dataset from 1 experiment
\item[\code{x}:] Object of class \code{"vector"} time or other independent
variable.  
\item[\code{nt}:] Object of class \code{"integer"} length \code{x}
\item[\code{x2}:] Object of class \code{"vector"}  vector of points in 
2nd independent dimension, such as wavelengths of wavenumbers
\item[\code{nl}:] Object of class \code{"integer"}  length \code{x2} 
\item[\code{C2}:] Object of class \code{"matrix"} concentration matrix for
simulated data
\item[\code{E2}:] Object of class \code{"matrix"} matrix of spectra for
simulated data
\item[\code{sigma}:] Object of class \code{"numeric"} noise level in 
simulated data 
\item[\code{mod\_type}:] Object of class \code{"character"} character string defining the model type, e.g., 
\code{"kin"} or \code{"spec"}
\item[\code{parnames}:] Object of class \code{"vector"} vector of
parameter names, used internally
\item[\code{finished}:] Object of class \code{"logical"} describes 
whether optimization is complete
\item[\code{simdata}:] Object of class \code{"logical"}  logical that is \code{TRUE} if the data is simulated, 
\code{FALSE} otherwise; will determine whether values in \code{C2} and 
\code{E2} are plotted with results
\item[\code{weightpar}:] Object of class \code{"list"}  list of vectors 
\code{c(first\_x, last\_x, first\_x2, last\_x2, weight)}, 
where each vector is of length 5 and 
specifies an interval in which to weight the data.  
\Itemize{
\item[first\_x] first(absolute, not an index) \code{x} to weight
\item[last\_x] last (absolute, not an index) \code{x} to weight 
\item[first\_x2] first (absolute, not an index)  \code{x2} to weight
\item[last\_x2] last (absolute, not an index)  \code{x2} to weight 
\item[weight] numeric by which to weight data 
}
Note that if vector elements 1-4 are \code{NA} (not a number), the firstmost 
point of the data is taken for elements 1 and 3, and the lastmost points
are taken for 2 and 4. 
For example, \code{ weight\_par = list(c(40, 1500, 400, 600, .9), 
  c(NA, NA, 700, 800, .1))}  will weight data between times 40 and 1500 
picoseconds and 700 and 800 wavelengths by .9, and will weight data at 
all times between wavelength 700 and 800 by .1.  
Note also that for single photon counting data 
\code{weightpar = list(poisson = TRUE)} will apply poisson weighting to all 
non-zero elements of the data. 
\item[\code{weight}:] Object of class \code{"logical"} \code{TRUE} when the specification in 
\code{weightpar} is to be applied and \code{FALSE} otherwise
\item[\code{weightM}:] Object of class \code{"matrix"} weights 
\item[\code{weightsmooth}:] Object of class \code{"list"} type of smoothing to apply with weighting; not currently
used
\item[\code{fixed}:] Object of class \code{"list"} list of lists or vectors giving the parameter values 
to fix (at their starting values) during optimization. 
\item[\code{clp0}:] Object of class \code{"list"}  list of lists with elements \code{low}, \code{high} and 
\code{comp}, specifying the least value in  \code{x2} to constrain 
to zero, the greatest value in  \code{x2} to 
constrain to zero, and the component to which to apply the zero constraint, 
respectively.  e.g., \code{clp0 = list(list(low=400, high = 600, comp=2), 
  list(low = 600, high = 650, comp=4))} applies zero constraints to the spectra 
associated with components 2 and 4. 
\item[\code{makeps}:] Object of class \code{"character"} 
specifyies the prefix of files written to 
postscript

\item[\code{clpequspec}:] Object of class \code{"list"} list of lists each of which has elements \code{to, 
  from, low, high}, and optional element \code{dataset} to specify the dataset
from which to get the reference clp (that is, a spectrum for kinetic 
models).  \code{to} is the component to be fixed in relation to some other 
component; from is the reference component.  \code{low} and \code{high} 
are the 
least and greatest absolute values of the \code{clp} vector to constrain. 
e.g., 
\code{clpequspec = list(list(low = 400, high = 600, to = 1, from = 2))} 
will constrain the first component to equality to the second component 
between wavelengths 400 and 600.  Note that equality constraints are
actually constraints to a linear relationship.  For each of the equality
constraints specfied as a list in the \code{clpequspec} list, specify a
starting value parameterizing this linear relation in the vector
\code{clpequ}; if true equality is desired then fix the corresponding
parameter in \code{clpequ} to 1.  Note that if multiple components are
constraints, the \code{from} in the sublists should be increasing order, 
(i.e., \code{(list(to=2, from=1, low=100, high=10000), 
list(to=3, from=1, low=10000, high=100))}, not \code{list(to=3, from=1, low=10000, high=100), 
list(to=2, from=1, low=10000, high=100)})
\item[\code{lclp0}:] Object of class \code{"logical"} \code{TRUE} if specification in \code{clp0} 
is to be applied and \code{FALSE} otherwise 
\item[\code{lclpequ}:] Object of class \code{"logical"} \code{TRUE} if specification in clpequspec 
is to be applied and \code{FALSE} otherwise
\item[\code{title}:] Object of class \code{"character"} displayed on output plots
\item[\code{mhist}:] Object of class \code{"list"}  list describing fitting history
\item[\code{datCall}:] Object of class \code{"list"} list of calls to functions
\item[drel] vector of starting parameters for dataset scaling relations
\item[\code{dscalspec}:] Object of class \code{"list"} 
\item[\code{drel}:] Object of class \code{"vector"}  vector of starting parameters for dataset scaling relations
\item[\code{clpequ}:] Object of class \code{"vector"}  describes the
parameters governing the clp equality constraints specified in \code{clpequspec}
\item[\code{scalx}:] Object of class \code{"numeric"}  numeric by which to scale the \code{x} axis in plotting
\item[prel] vector of starting values for the relations described in 
prelspec
\item[\code{prel}:] Object of class \code{"vector"}  vector of starting values for the relations described in 
prelspec
\item[\code{prelspec}:] Object of class \code{"list"}  list of lists to specify the functional
relationship between parameters, each of which has elements 
\Itemize{ 
\item[what1] character string describing the parameter type to relate, 
e.g., \code{"kinpar"} 
\item[what2] the parameter type on which the relation is based; usually 
the same as \code{what1}
\item[ind1] index into \code{what1}
\item[ind2] index into \code{what2}
\item[rel] character string, 
optional argument to specify functional relation type, 
by default linear }
e.g., 
\code{prelspec = list(list(what1 = "kinpar", what2 = "kinpar", ind1 = 1, 
  ind2 = 5))}  relates the 1st element of \code{kinpar} to the 5th element of 
\code{kinpar}.  The starting values parameterizing the relationship are 
given in the \code{prel} vector
\item[\code{fvecind}:] Object of class \code{"vector"}  vector containing indices of fixed parameters 
\item[\code{pvecind}:] Object of class \code{"vector"}  used internally to
store indices of related parameters. 
\item[\code{groups}:] Object of class \code{"list"}  list containing lists of pairs c(x2 index, dataset index). 
the \code{x2} values (which are solved for as conditionally linear 
parameters) are equated for all pairs in a list.
\item[\code{iter}:] Object of class \code{"numeric"}  describing the 
number of iterations that is run; this is sometimes 
stored after fitting, but has not effect as an argument to  
\code{\LinkA{initModel}{initModel}}
\item[\code{clpCon}:] Object of class \code{"list"}  used internally to enforce constraints on the clp
\item[\code{ncomp}:] Object of class \code{"numeric"}  describing the number of components in a model
\item[\code{clpdep}:] Object of class \code{"logical"}  describing whether a model is dependent on the index
of \code{x2}
\item[\code{inten}:] Object of class \code{"matrix"}  for use with FLIM data; represents the number of photons
per pixel measured over the course of all 
times $t$ represented by the dataset.  See the help for the \code{readData}
function for more information. 
\item[\code{positivepar}:] Object of class \code{"vector"}  containing
character strings of those parameter vectors to constrain to positivity,
e.g., \code{positivepar=c("kinpar")}
}
\end{Section}
\begin{Author}\relax
Katharine M. Mullen, Ivo H. M. van Stokkum
\end{Author}
\begin{SeeAlso}\relax
\code{\LinkA{kin-class}{kin.Rdash.class}}, \code{\LinkA{spec-class}{spec.Rdash.class}}
\end{SeeAlso}
\begin{Examples}
\begin{ExampleCode}
# simulate data 

 C <- matrix(nrow = 51, ncol = 2)
 k <- c(.5, 1)
 t <- seq(0, 2, by = 2/50)
 C[, 1] <- exp( - k[1] * t)
 C[, 2] <- exp( - k[2] * t) 
 E <- matrix(nrow = 51, ncol = 2)
 wavenum <- seq(18000, 28000, by=200)
 location <- c(25000, 20000)
 delta <- c(5000, 7000)
 amp <- c(1, 2)
 E[, 1] <- amp[1] * exp( - log(2) * (2 * (wavenum - location[1])/delta[1])^2)
 E[, 2] <- amp[2] * exp( - log(2) * (2 * (wavenum - location[2])/delta[2])^2)
 sigma <- .001
 Psi_q  <- C 

 # initialize an object of class dat 
 Psi_q_data <- dat(psi.df = Psi_q, x = t, nt = length(t), 
 x2 = wavenum, nl = length(wavenum))

 # initialize an object of class dat via initModel 
 # this dat object is also a kin object
 kinetic_model <- initModel(mod_type = "kin", seqmod = FALSE, 
 kinpar = c(.1, 2))
\end{ExampleCode}
\end{Examples}

